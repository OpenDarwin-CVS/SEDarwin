\section{The Darwin BSD Subsystem}

The SEDarwin MAC Framework currently provides label storage and
access mediation for processes and files, as well as System V, POSIX
and Mach IPC objects.

\subsection{Process Labels}

Like the MAC Framework on FreeBSD, the Darwin MAC Framework adds label
storage to both the process structure, {\tt struct proc} and the user
credential structure, {\tt struct ucred}.  Since the shared process
credential structure includes a reference to the user credentials,
both labels are available anytime a BSD process is available.  Most
policy modules (including SEDarwin) store labels in the shared user
credential structure, but some policies may require storage space
directly in the process structure.  One example of the latter
approach is the LOMAC policy\cite{Fraser00,fraser01}).

\subsection{Process Controls}

The Darwin MAC Framework provides entry points for the same four process
control checks that the FreeBSD MAC framework provides:
\footnote{On these, only {\tt mac\_check\_proc\_signal()} is presently
implemented by the SEDarwin module.}

\begin{center}
\begin{tabular}{ll}
{\tt mac\_check\_proc\_signal()} & {\tt mac\_check\_proc\_sched()} \\
{\tt mac\_check\_proc\_debug()} & {\tt mac\_check\_proc\_wait()} \\
\end{tabular}
\end{center}

The Darwin MAC Framework also provides an additional six process
control checks not currently present in the FreeBSD MAC framework:
\footnote{The audit-related entry points are not currently
implemented for the SEDarwin module.}

\begin{center}
\begin{tabular}{ll}
{\tt mac\_check\_proc\_getaudit()} & {\tt mac\_check\_proc\_setaudit()} \\
{\tt mac\_check\_proc\_getauid()} & {\tt mac\_check\_proc\_setauid()} \\
{\tt mac\_check\_proc\_getlcid()} & {\tt mac\_check\_proc\_setlcid()} \\
\end{tabular}
\end{center}

\subsection{File Labels}

The Darwin MAC Framework adds label storage to vnodes ({\tt struct
vnode}), devfs device nodes ({\tt struct devnode}), and the mount
structure ({\tt struct mount}).  The vnode is the core structure
for all file activity.  A vnode is described by {\tt struct vnode}.
There is a unique vnode allocated for each active file, each current
directory, each mounted-on file, text file, and the root.  As with
the FreeBSD MAC Framework, the mount structure contains two labels,
one to label the mount point itself and a second label that may be
used as the default label on single-label file systems.

Policy modules may use HFS+ extended attributes to provide persistence
for file labels, or the policies may implement their own non-persistent
file labeling scheme.  For example, SEDarwin labels devfs and other
non-disk file system entries based on rules in the Type Enforcement
policy.

\subsection{File System Controls}

The Darwin MAC Framework provides file access control entry points
with semantics equivalent to the FreeBSD MAC Framework.  The complete
set of file entry points could not be carried over from FreeBSD due
to minor differences between the two operating systems (FreeBSD has
introduced a small number of additional file APIs that must be
protected).  The current set of entry points is:\footnote{The
extended attribute and file attribute list entry points are not
currently implemented in the SEDarwin policy module.}

\begin{center}
\begin{tabular}{ll}
{\tt mac\_check\_vnode\_access()}
& {\tt mac\_check\_vnode\_chdir()} \\

{\tt mac\_check\_vnode\_chroot()}
& {\tt mac\_check\_vnode\_create()} \\

{\tt mac\_check\_vnode\_delete()}
& {\tt mac\_check\_vnode\_deleteextattr()} \\

{\tt mac\_check\_vnode\_exchangedata()}
& {\tt mac\_check\_vnode\_exec()} \\

{\tt mac\_check\_vnode\_getattrlist()}
& {\tt mac\_check\_vnode\_getextattr()} \\

{\tt mac\_check\_vnode\_link()}
& {\tt mac\_check\_vnode\_listextattr()} \\

{\tt mac\_check\_vnode\_lookup()}
& {\tt mac\_check\_vnode\_mmap()} \\

{\tt mac\_check\_vnode\_mprotect()}
& {\tt mac\_check\_vnode\_open()} \\

{\tt mac\_check\_vnode\_poll()}
& {\tt mac\_check\_vnode\_read()} \\

{\tt mac\_check\_vnode\_readdir()}
& {\tt mac\_check\_vnode\_readlink()} \\

{\tt mac\_check\_vnode\_rename\_from()}
& {\tt mac\_check\_vnode\_rename\_to()} \\

{\tt mac\_check\_vnode\_revoke()}
& {\tt mac\_check\_vnode\_select()} \\

{\tt mac\_check\_vnode\_setattrlist()}
& {\tt mac\_check\_vnode\_setextattr()} \\

{\tt mac\_check\_vnode\_setflags()}
& {\tt mac\_check\_vnode\_setmode()} \\

{\tt mac\_check\_vnode\_setowner()}
& {\tt mac\_check\_vnode\_setutimes()} \\

{\tt mac\_check\_vnode\_stat()}
& {\tt mac\_check\_vnode\_write()} \\

\end{tabular}
\end{center}

Darwin adds several new VFS operations to support the HFS+ file system
to provide aggregate vnode data to user space or to increase system
performance.  For instance, the {\tt readdirattr} operation returns
attributes for all files in the specified directory.  The {\tt
readdirattr} operation uses optimized routines within the HFS+ file
system implementation to avoid the performance penalty associated with
creating a vnode for each file in the directory.  The current design
of the MAC Framework does not address aggregate operations, nor does
it address how to best filter the data returned to the user while
avoiding increased performance costs.  The {\tt getattrlist()} and {\tt
setattrlist()} system calls operate on HFS+ specific file attributes,
such as type and creator.
\footnote{As the {\tt readdirattr} and {\tt setattrlist} VFS operations
are not required for proper operation of the system they may be
disabled in a future version of the prototype.}

In this release of SEDarwin, a small number of HFS-specific operations
have not yet been secured by the MAC Framework.  These include {\tt
copyfile}, {\tt searchfs}, and {\tt undelete}.
