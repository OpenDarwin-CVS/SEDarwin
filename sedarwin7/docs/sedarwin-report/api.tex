\section{Application Interfaces}

\subsection{MAC Framework APIs}

The MAC Framework on Darwin provides a number of system calls and system
controls (sysctls) to improve the functionality of security-aware
applications, including label management tools and the login routines.
These routines generally operate in a policy-agnostic manner, allowing
applications to deal with labels and elements in an abstract manner.

\subsubsection{Extended Attribute APIs}

A prototype implementation of extended attributes was developed for
use with file security labels. The extended attribute support is
also available to user programs, through an interface compatible
with FreeBSD 5. \footnote{Mac OS X 10.4 (Tiger) includes an
independently-developed extended attribute implementation that uses
a different API.} A file may have multiple extended attributes,
each identified by a namespace and a name. There are only two
namespaces, system and user.  An unprivileged user process may not
examine or change system extended attributes, but any user with
permission on the file may change the user attributes.  Since the
MAC Framework supports multiple simultaneous policies, each with
their own file label formats, multiple attributes per file may be
required. Because the file labels stored in extended attributes can
be changed with the extended attribute manipulation calls as well
as the normal label manipulation calls, all security labels should
use the ``system'' attribute namespace to prevent unprivileged users
from changing the labels directly.  We have implemented system calls
for reading, writing, deleting and listing extended attributes.
However, the system calls to delete and list extended attributes
have not yet been implemented for the HFS+ filesystem.  Any use of
the system attribute namespace requires that the caller be privileged.
Since security policies may have their own privilege requirements,
this is not the same as ``root''. The user attribute data is protected
according to the normal discretionary and mandatory protections
associated with the file or directory.

\begin{longtable}{|p{1.5in}|p{3in}|}
\hline
{\bf System Call} & {\bf Purpose} \\
\hline
\hline
\endfirsthead
\hline
{\bf System Call} & {\bf Purpose} \\
\hline
\hline
\endhead
\hline
\endfoot
extattr\_set\_file, extattr\_set\_fd, extattr\_set\_link & Set the value of an extended attribute. \\[3mm]
\hline
extattr\_get\_file, extattr\_get\_fd, extattr\_get\_link & Retrieve the value of an extended attribute. \\[3mm]
\hline
extattr\_delete\_file, extattr\_delete\_fd, extattr\_delete\_link & Remove an extented attribute; currently unimplemented \\[3mm]
\hline
extattr\_list\_file, extattr\_list\_fd, extattr\_list\_link & List the extented attributes for a file; currently unimplemented \\[3mm]
\hline
extattrctl & Manage extended attributes; currently unimplemented \\
\end{longtable}

\subsubsection{Label Management}
Many security policies that use persistent object labels (such as type
enforcement or MLS) require a system administrator or security officer to
set or change labels. Label manipulation tools must use the new
MAC Framework interfaces for reading or writing
labels from user processes. This interface understands that there may be
multiple security policies in use, and that each policy may have different
internal and external formats. Currently, the external (user space) format
for labels must be in the form of a text string. In a typical read label
call, the user provides a reference to the object (such as a file descriptor),
and the names of policies to retrieve the associated labels.

\begin{longtable}{|p{1.8in}|p{2.2in}|}
\hline
{\bf System Call} & {\bf Purpose} \\
\hline
\hline
\endfirsthead
\hline
{\bf System Call} & {\bf Purpose} \\
\hline
\hline
\endhead
\hline
\endfoot
{\tt mac\_set\_file()}, {\tt mac\_set\_fd()}, {\tt mac\_set\_link()} & Change the label associated with a file. \\[3mm]
\hline
{\tt mac\_get\_file()}, {\tt mac\_get\_fd()}, {\tt mac\_get\_link()} & Retrieve the label associated with a file. \\[3mm]
\hline
{\tt mac\_set\_proc()} & Change the label of the current process. \\[3mm]
\hline
{\tt mac\_get\_pid()}, {\tt mac\_get\_proc()} & Retrieve the label on a process. \\[3mm]
\hline
{\tt mach\_get\_label\_text()} & Retrieve the label on a port as a label handle. \\[3mm]
{\tt mach\_get\_label\_text()} & Retrieve the label on a port or label handle in text form. \\[3mm]
\hline
{\tt mach\_set\_port\_label()} & Change the label on a port. \\[3mm]
\hline
{\tt mach\_get\_task\_label\_text()} & Retrieve the label on a task as a label handle. \\[3mm]
{\tt mach\_get\_task\_label\_text()} & Retrieve the label on a task in text form. \\[3mm]
\hline
{\tt mac\_label\_new()} & Create a new label handle. \\[3mm]
\hline
{\tt mac\_request\_label()} & Compute a new label based on subject, object, service name. \\
\end{longtable}

\subsubsection{Secure Program Execution}
Security-aware programs may require a mechanism for executing a
program with an arbitrary label.  This functionality is also useful
for testing new security policies.  We provide the {\tt mac\_execve()}
system call for this purpose. It extends the standard UNIX {\tt
execve()} system call with an additional argument, the label to use
when the new program is executed.  By using this new API, the caller
is assured that the label change will be atomic with respect to
program execution.  With a normal {\tt execve()} call followed by
a label change, this may not be the case.

\subsubsection{Extensible Policy Interfaces}
Large user space services (such as window systems) may serve multiple
users or security domains from one process. Because of this, the
kernel alone cannot be responsible for securing such a process. The
kernel can assist a multiuser service in controlling access by
providing policy entry points that allow a user space program to store
its own labels, and request access decisions from the kernel when
necessary. For example, a window system could maintain a label for
each connection and for each window. The Framework entry point we have
chosen to implement in our prototype simply passes a subject, an
object, a service name, and a permission name, to each configured
security policy, and returns a decision (composed in the same way
as an internal access check). The subject and object provided to
the kernel may be label handles, allowing the program to store
labels internally. With the SEDarwin policy module, the policy can
define new service and permission names for use by user space programs.
Continuing the window system example, each attempt by a connection
to read the contents of a window could be controlled by a call
similar to the following:

\begin{verbatim}
mac_check_port_access(connection_label, window_label,
    "window", read")
\end{verbatim}

Policies may also need unstructured interfaces to user space for
management, configuration, or auditing purposes. We provide the
{\tt mac\_syscall()} interface for this.  The caller provides the
Framework policy name and a block of arbitrary data. Only the named
policy receives the call.

\subsection{User Space Components}

A small number of Darwin binaries were modified to increase security
and to improve the user exeperience.  Two programs were modified
to be label-aware ({\tt ls} and {\tt ps}) and several more were
added to help users manage labels ({\tt getfmac}, {\tt getpmac},
{\tt setfsmac}, {\tt setpmac}).  In addition, the {\tt lcs} command
was added to provide a way to show login context information.  Other
Darwin programs were modified to help the MAC Framework manage
policies and increase system security: {\tt mach\_init}, {\tt MiG},
and {\tt loginwindow}).

\subsubsection{Mach Namespace}

The {\tt mach\_init} program was our testbsd for user space Mach
security work. {\tt mach\_init} was chosen because it is open source,
provides a service that needs to be secured (the Mach port namespace),
and is relatively simple.  Unlike in a traditional Mach system, in
Darwin the bootstrap namespace is used during the entire system
operation and is not replaced by a different server after the system
boots. The namespace service is accessed by MiG calls to a namespace
port, one of which is stored in every task's list of registered
ports (a child task inherits the set of registered ports from its
parent). {\tt mach\_init} sets the bootstrap port to one it created
to handle namespace requests, so all new processes share the same
namespace (unless overridden by the process or one of its ancestors).
User programs can register server ports in the namespace, look up
existing registered ports, and create new namespaces. A new namespace
inherits the contents of the processes' old namespace, but further
changes to it do not affect the original. When a new namespace is
created, the caller can overwrite the old namespace port with the
new one, so that child processes will inherit the new namespace.
In a Mac OS X system, the {\tt loginwindow} process creates a new
namespace for the user's login session.  {\tt loginwindow} also
arranges for processes created for the user's login session to have
the new namespace port registered instead of the original one.  Some
policies may use different labels for login session namespaces and
the system namespace for additional protection.

{\tt mach\_init} was rebuilt with our security-enhanced MiG to
enforce permission checks in the bootstrap namespace(s). This
mechanism uses the "access decision" trailer. The sender's label
may also be used for computing labels on new namespaces or other
objects but this is not currently done.

However, in implementing the use of additional message trailers on
Darwin 7.8 (Mac OS X 10.3.8), unexpected behaviors were observed
when attempting to request extended-length message trailers early
on in the system boot process. The system would seemingly deadlock
when trailers over a certain length were requested from the kernel.
After much debugging and conversation with Apple personnel, it was
determined that our use of extended trailers in mach\_init had
uncovered a bug in the vendor provided sources. We were informed
by Apple that this behavior was not present in Mac OS X 10.4 (Tiger),
and advised to use 10.4 instead.  No fix has been provided for Mac
OS X 10.3 (Panther) and we were unable to discern the fix due to
the large number of changes between 10.3 and 10.4.

We plan to secure launchd (which replaces functionality previously
provided by both mach\_init and the BSD init) once we move forward
with porting the SEDarwin codebase to 10.4.  We have independently
verified that adding extended trailers to launchd on a stock 10.4
system does work as expected.

\subsubsection{Login}
The Mac OS X {\tt loginwindow} process is responsible for displaying
the login window at system startup, which reads user names and
passwords.  {\tt loginwindow} works in conjunction with the {\tt
WindowServer} process; login operations are divided between the two
in order to support fast user switching.  {\tt loginwindow} supports
a plug-in interface that can be used to implement multi-stage user
authentication.  We have implemented a loginwindow plug-in for use
with the MAC Framework that runs after the user has authenticated
but before any processes are run on behalf of the user.  Our plug-in
supports policy-specific modules via its own plug-in framework.

The SEDarwin MAC login plug-in queries the running FLASK security
policy and, if the policy specifies one or more roles for the user,
presents the user with a list of available roles.  The plug-in then
updates the label of the processes specified in the {\tt
/etc/MAClogin.conf} file (by default {\tt WindowServer} and {\tt
ATServer}) to match the selected role.  If no role exists for the
user, they will be unable to log in.  Upon logout, the labels on
the affected processes are restored.

For network and non-graphical console logins, a PAM module is used
to create the login context.  A separate policy-specific PAM module
is then called to store the the actual label.  This module may also
be used to present the user with a choice of roles if the policy
lists more than one role for that user.

In the default Mac OS X configuration, the system will auto-login
a particular user on the console without displaying the login window.
This is clearly unacceptable for SEDarwin and as such, auto-login
is disabled as part of the installation process.  Fast user switching
is also disabled during installation.

\subsubsection{Label Commands}
The basic commands for listing files and processes, {\tt ps} and {\tt ls},
were augmented to display security labels. As on FreeBSD and SELinux, the {\tt -Z} option
to either program causes it to display security labels for the policies listed
in {\tt /etc/mac.conf}. As the MAC Framework uses ASCII text strings for
label input and output from user space, there is no need for any display
conversion.

SELinux, SEBSD, and now SEDarwin use a similar file labeling tool, which
uses regular expressions matching file names for user convenience. 
