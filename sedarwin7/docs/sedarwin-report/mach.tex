\section{The Darwin Mach Subsystem}

\subsection{Introduction}

In securing the Mach operating system, there are three major areas
to be covered: traps, messages and servers.

The entry point for Mach kernel services from user space is the
trap, or system call. Mach provides very few traps. Most services
are implemented as Mach servers and accessed via Mach messages (all
of which use the same trap).  Darwin contains a small number of
traps for performance-sensitive operations, such as operating on
semaphores or reading timers.  However, those services are unlikely
to require mediation by a policy, and are not mediated in our current
implementation. The remaining services are accessed via messaging,
and as such can be treated the same as user space services for
security purposes.

The IPC mechanism provided by Mach is message passing: a task sends
a message to a port, which may be received by another task, including
the kernel. A port is the Mach name for a message receipt queue.
Mach messages are structured, they can contain several types of
data specially interpreted by the kernel in addition to raw data.
Messages can send port rights, allowing another task to have access
to a port. Either a send right, a send-once right, or a receive
right can be transferred in this way. Address spaces can also be
shared via messaging. The receiving task gets a mapping for the
memory and can access it directly, without requiring extra copies
as in unstructured IPC systems.

The Mach operating system provides other services, such as device
abstraction, virtual memory, scheduling, etc., but those services are
accessed via messages sent to kernel ports. So, the other services
provided by the Mach kernel can be treated as user space servers for
security purposes.

Mach messages contain a variable number of elements, each of which may be
raw data, port names, or memory ranges. The kernel processes the port name and
memory range sections specially, allowing the port names to be used by the
receiving task.

\subsubsection{Mach Interface Generator (MiG)}

The Mach Interface Generator (MiG) is similar to rpcgen on UNIX
systems. MiG reads a specification of data types and method prototypes
and generates client and server stubs which copy the arguments and a
method ID into a buffer and send it as a Mach message.  The caller
must provide the name of a port that will receive the message.  Since
Mach ports are uni-directional, the return value and any reference
parameters must be sent to the client in a separate Mach message. The
MiG-generated stubs handle this interaction--the client is not
concerned with the actual semantics of message passing or port rights
and may simply make library calls to perform IPC/RPC tasks.

Certain ``special'' kinds of message contents are also supported by
MiG. Port rights and shared memory can be passed as both calling
parameters and return values and large arrays are automatically
passed as copy-on-write shared memory. Full information about
standard MiG types supported in Darwin can be found in the {\tt
/usr/include/mach/std\_types.defs} and {\tt
/usr/include/mach/mach\_types.defs} definition files.

The model of ``one object, one port'' works well with MiG. The
first argument specifies the destination port for the message that
the function call produces. This port represents the object on which
the function operates.  For example, a window server may contain a
method to resize a window:

\begin{verbatim}
    routine window_resize(window : mach_port_t;
                          width  : int;
                          height : int);
\end{verbatim}

In this example the message is sent to the port specified as the
first argument.  When the window server receives the message, the
MiG stub can convert the port into an implementation-specific pointer
through a translation type. MiG removes the need for application
developers to implement wire protocols in most cases.

A collection of routines that will all be sent to the same type of
port (e.g. port for which a window server hold a receive right) are
called a subsystem. MiG only allows a subsystem to contain up to
100 methods. New methods can be added to an existing subsystem
without affecting existing clients as long as the new total number
of methods remains less than or equal to 100.

User programs linked with client stubs can call the client methods
which will function ``as if'' the remote method existed in the client
task's address space. A server task (or part of the kernel)
implementing a subsystem uses the server stubs to unpack the messages
containing the method IDs and arguments.

Both kernel and user space services make use of MiG. For example the
standard Mach subsystems for virtual memory, devices, and task
management all use MiG-format messages. On Darwin, many user space
systems, such as the namespace server, notification server, and
pasteboard server use MiG-formatted messages as well.  Using MiG
whenever practical greatly reduces the development overhead of
implementing messaging protocols. This is especially pertinent when
developing kernel servers since the MiG generated code will handle
all of the necessary {\tt copyin()}, {\tt copyout()}, and basic message
validation tasks.


\subsection{Mach Protection Primitives}

A security policy may want to attach a security label to
tasks, messages, and ports. DTOS stores a security identifier in
kernel objects and adds secured versions of Mach messaging routines
which return or require a security identifier. Our model is similar but,
because we support multiple policies, we use a policy-agnostic label
structure instead of a simple security identifier. This label
structure is added to every port (ipc\_port\_t) and task (task\_t).
The convention of using a port to represent a single object means
that labels on ports translate well into labels on objects for
policy purposes. We do not currently label the messages themselves,
nor do we currently label virtual memory objects.

The policy-agnostic label structure presents some implementation
challenges compared to systems that assume a particular (often small
or at least fixed-size) label format. Because policies differ in
their label formats, the policy itself must handle label memory
management.  This means that copying a label is a potentially
expensive operation compared to copying a security identifier. In
addition, the MAC Framework does not currently permit policies
to precompute security decisions for later use as many common
policies (type enforcement, MLS, capability models, etc) might
otherwise do.  Policies are free, however, to implement their own
caching mechanisms to speed up label computations or other operations.
The access vector cache used by FLASK is one example of this.

\textit{// XXX: This doesn't really belong here...I updated to our text, but
where should it go?}

Because the Darwin kernel includes a BSD subsystem as well, the
security additions to both Mach and BSD components will need to
remain correlated. For example, a BSD process and its underlying
Mach task should have the same subject label(s). This is accomplished
by augmenting the routines in the BSD subsystem to update the task
labels whenever a process label is changed.  Due to locking
constraints, Mach system calls or message handlers cannot acquire
BSD-side locks, and so are unable to change the BSD process label.
In order to ensure that the system does not deadlock due to this,
no user space interface is provided to directly modify task label
information.


\subsection{Mach Labels}

The additions for the Mach portion of the Darwin kernel all follow a
similar model to the MAC Framework implementation. Labeling
operations are inserted into object creation and deallocation
routines, and access decision calls are inserted into object usage
routines.

Access control checks have been added to the Darwin Framework for
message sending and receiving, port right transfers (both send and
receive) as well as relabel operations for ports and tasks.  The
full list of access control entry points is as follows.

\begin{longtable}{|p{2.2in}|p{2.3in}|}
\hline
{\bf Entry Point} & {\bf Called When} \\
\hline
\hline
\endfirsthead
\hline
{\bf Entry Point} & {\bf Called When} \\
\hline
\hline
\endhead
\hline
\endfoot
{\tt mac\_check\_port\_send()} & task attempts to send a message \\[3mm]
\hline
{\tt mac\_check\_port\_receive()} & task attempts to receive a message \\[3mm]
\hline
{\tt mac\_check\_port\_make\_send()} & task attempts to create a send right from a receive right \\[3mm]
{\tt mac\_check\_port\_make\_send\_once()} & task attempts to create a send-once right from a receive right \\[3mm]
\hline
{\tt mac\_check\_port\_copy\_send()} & task attempts to copy a send right to a task \\[3mm]
{\tt mac\_check\_port\_move\_send()} & task attempts to move a send right from one task to another \\[3mm]
{\tt mac\_check\_port\_move\_send\_once()} & task attempts to move a send-one right from one task to another \\[3mm]
\hline
{\tt mac\_check\_port\_move\_recieve()} & task attempts to move the receive right from one task another. \\[3mm]
\hline
{\tt mac\_check\_port\_hold\_send()} & task attempts to receive a send right \\[3mm]
{\tt mac\_check\_port\_hold\_send\_once()} & task attempts to receive a send-once right \\[3mm]
\hline
{\tt mac\_check\_port\_hold\_recieve()} & task attempts to receive a receive right \\[3mm]
\hline
{\tt mac\_check\_port\_relabel()} & task attempts to relabel a port or task \\[3mm]
\end{longtable}

\subsection{Changes to the Mach Kernel}

Kernel changes made to support the MAC Framework were fairly
noninvasive, only affecting a few of the data structures and a small
number of functions.  Tasks and ports received label storage.
Creation and deallocation routines call labeling event operations.
Tasks use label handles because we can use the lock built in to the
label handle to protect the label instead of the task lock. The task
lock is at a position in the established lock order that precludes its
use in security checks (which are usually deeply nested).  Another
advantage of using a label handle reference to store the task label is
that when a task requests the labels trailer (described in the next
section), only a port right copyout is required to store a reference
to the sending task's label.  Using a new label would require
allocating storage for the label, making a copy of the sending task's
label, and a copyout to the caller's task, which is much more
expensive.

Changes to the Darwin BSD subsystem were also necessary.  Two separate
labels are required because the Mach part of the kernel cannot acquire
BSD-side locks.  However, labels for Mach tasks and BSD processes need
to be kept synchronized, as there really is only one subject (the Mach
task\_t and the BSD struct proc both include an opaque pointer the
other).

A task always has an associated task port, used for access control.
This port also has its own label.  Our system updates the task port
label whenever the task label itself is changed.  Changes to the
task port label are not reflected by the task label.  This makes it
possible for the task label and the task port label to diverge if
desired (and permitted by the policies).

\subsubsection{Mach security server}

A new kernel-resident Mach server has been added to handle security-specific
requests.  As with the other user-accessible servers in a Mach
system, the security server is implemented as a MiG subsystem and
may be called from either user or kernel space.  The MiG definition
file for the server may be found in the {\tt osfmk/mach/security.defs}
file in the SEDarwin kernel sources.  The functions provided by the
security server fall into two basic categories: label operations
and access control checks.

\subsubsection{Message trailers}

Darwin has added a trailer component to the traditional Mach message
that is located after the end of message body.  Message trailers
are used by the Darwin Mach subsystem to append additional (out of
line) information about the sender to a message when it is delivered
if requested by the receiver.  In a stock Darwin system the recipient
can request three trailer fields: a message sequence number, a
security token (equivalent to the UNIX uid and gid), and an audit
token (an opaque object which may be interpreted by the BSM audit
library).  We have added two additional trailer types:
{\tt MACH\_RCV\_TRAILER\_LABELS} and {\tt MACH\_RCV\_TRAILER\_AV}.

The {\tt MACH\_RCV\_TRAILER\_LABELS} trailer returns the sender's
security labels as part of the message trailer in the form of a
label handle.  The caller may use the label handle to request access
decisions from the security server or convert the handle to text
form.  For example, the bootstrap namespace server could ensure
only a privleged process can examine the namespace created by a
different loginwindow process.  The returned label handle may be
deallocated using the standard Mach port API functions {\tt
mach\_port\_deallocate()} and {\tt mach\_port\_destroy()}.  Label
handles are garbage collected upon task termination just like regular
Mach ports.

The {\tt MACH\_RCV\_TRAILER\_AV} trailer trailer provides similar
functionality to the DTOS {\tt mach\_msg\_secure()} trap without
requiring an extra trap (and the requisite changes to the standard
C library).  This trailer provides a single access decision as a
boolean value to the caller, computed by the loaded policies using
subject, port, subsystem, and routine name arguments  Providing the
full subsystem vector (as in DTOS) is impractical with our abstract
policy framework.  A complex policy would be required to compute
decisions for the entire subsystem even though only one decision
was useful at that time and the results of computation may not be
usable for future calls.  An alternative is to cache access decisions
in the policy itself for rapid lookup, as FLASK does.  Our extended
trailers may be use alone or in combination with any other trailers
supported by Darwin.

When the extended trailers are requested by a receiving task, calls
are made to the loaded security policies and the result is copied
out. Composition of these decisions is performed in the same manner
as all other access controls managed by the MAC Framework: an
operation is allowed only if all individual policies allow it.
Although we only report a single access control decision in a
trailer, we have not precluded a DTOS-style optimization where
precomputed access control decisions are cache in the ipc\_entry
field of the sending task.

Unlike the stock Darwin trailer fields, our additional trailers are
not simply existing data that can be copied.  The kernel must make
additional calls to the MAC Framework to fill in the trailer data.
Because of this, the label handles are only placed in the trailer
when requested by the recipient. Prior to the introduction of label
handles, we stored a string version of the sending task's label in
the message trailer.  This solution proved to be unworkable because
the strings could be of arbitrary length, and repeated conversions
to and from string representation was prohibitively expensive.

\textit{XXX: Again, not sure where this next bit belongs}

MiG also handles providing trailer contents to the user; method argument definitions can
have a flag indicating that data will come from the appropriate part of the trailer instead
of from the client. We extended MiG to handle the labels in the trailer; the flag
{\tt servermsglabels} on an argument of type {\tt msg\_labels\_t} supplies the labels.
\begin{samepage}
As an example the definition of the {\tt bootstrap\_create\_server} method implemented by
{\tt mach\_init} is shown:
\begin{verbatim}
routine bootstrap_create_server(
                 bootstrap_port : mach_port_t;
                 server_cmd     : cmd_t;
                 server_uid     : integer_t;
                 on_demand      : boolean_t;
 ServerSecToken  token          : security_token_t;
 servermsglabels labels         : msg_labels_t;
 out             server_port    : mach_port_make_send_t);
\end{verbatim}
\end{samepage}

The {\tt token} and {\tt labels} arguments come from the message trailer.

\subsection{Application/Server Security Techniques}

The flexibility of Mach IPC and the associated MiG message format provide a number
of ways for a multiuser server to provide access controls on the data it manages.
One common model for UNIX server programs consists of a master process and several
session processes. The master process waits for new connections and forks a session process
for each. The session process runs in the context of the authenticated user; the master
process usually has to run as {\tt root} for the privilege of changing user contexts.
This allows the operating system's normal access control checks to be used by the server;
the server only needs to handle the authentication.  However, using multiple session processes may
be inefficient, especially if communication with the master is required.

This model works
best for servers dealing primarily with file system resources. On systems where threads of the
same process can have different security contexts, per-session threads could be used instead
of processes for simpler communication (but more risk). In systems with structured
communication, such as Mach IPC, the security decisions can be separated from operation,
but the user's request is still available to the security routines. One way to achieve
such decoupling is to use a proxy server for security decisions. The proxy server reads
the same message formats as the application; after decoding the arguments, the proxy
server checks authorization with the security system. Depending on the result of the
authorization check, the proxy server calls the application with the user's message.


\subsection{Sequence of Messaging Security Checks}

Message security checks are performed at both the sending and
receiving ends.  When sending, the kernel first calls the
{\tt mac\_check\_port\_send()} Framework entry point to permit or deny
the sending of the message.  Next, each security policy is queried
for permission to transfer each port right contained in the message.
If any port right transfers are denied, the message is not sent.
Because memory objects are not currently labeled, no special check
is performed for transferring out-of-line memory.  The sending
checks refer to the receiving port, not the receiving task, as the
receive right for the destination port could be in the process of
moving between tasks while the message is in-flight.  On the receiving
end, the kernel calls the {\tt mac\_check\_port\_receive()} Framework
entry point to permit or deny receipt of the message.  If the task
is permitted to receive the message, further checks are made to
verify that the task is permitted to hold the port rights contained
in the message.  If any port right transfers are denied, the message
will not be delivered.

\subsubsection{Message Life Cycle}

\begin{enumerate}
\item {Message sent by task.}
The {\tt mach\_msg()} or {\tt mach\_msg\_send()} library function enters the
kernel via the {\tt mach\_msg\_trap()} syscall.

\item {Kernel allocates space for the (kernel-private) message object.}
The {\tt mach\_msg\_trap()} syscall calls {\tt mach\_msg\_send()} which uses
{\tt ipc\_kmsg\_get()} to allocate a new kernel message object.

\item {Framework entry point {\tt mac\_check\_port\_send()} is called.}
The {\tt mach\_msg\_send()} function calls {\tt ipc\_kmsg\_copyin()}
to copy the message from user space into the kernel.  The message header
and an optional message body are copied separately.  Before
copying the header in {\tt ipc\_kmsg\_copyin\_header()}, the
{\tt mac\_check\_port\_send()} Framework entry point is called to check
whether the task is permitted to send the message.

\item {The message header is filled in from user space by {\tt ipc\_kmsg\_copyin\_header()}.}
If the header contains port rights, {\tt ipc\_right\_copyin()} is called
to update the capabilities from the user's task.  Before port
rights are updated, the appropriate Framework entry point is
called.  The entry points are {\tt mac\_check\_port\_make\_send()} for
new send (or send-once) rights, {\tt mac\_check\_port\_move\_receive()}
for receive rights, and {\tt mac\_check\_port\_copy\_send()} for copying
or moving send (or send-once) rights.

\item {The message body (if any) is copied by {\tt ipc\_kmsg\_copyin\_body()}.}
The message body may contain additional port rights and inline
or out of line data (which may itself contain port rights).
Port rights (and their permission checks) are performed in the
same manner as the rights contained in the message header.
Space for memory regions (both inline and out of line) is
allocated from the kernel's ipc copy memory map (using copy on
write semantics).

\item {The message object is placed in the port's message queue.}
The {\tt mach\_msg\_send()} function calls {\tt ipc\_kmsg\_send()} which in
turn uses {\tt ipc\_mqueue\_send()} to enqueue the message.

\item {Message sent.}

....

\item {Receiving task tries to pick up the message.}
The {\tt mach\_msg()} or {\tt mach\_msg\_receive()} library function enters
the kernel via the {\tt mach\_msg\_trap()} syscall.

\item {Framework entry point {\tt mac\_check\_port\_receive()} is called.}
{\tt mach\_msg\_receive()} calls {\tt ipc\_mqueue\_receive()} which in turn
calls the Framework entry point {\tt mac\_check\_port\_receive()} to check
whether the task is permitted to receive the message.  If the task
is permitted to receive it, the message is moved out of the receiving
port's message queue.  If permission is denied, the message is dropped
and the in-kernel message object is deallocated.

\item {MAC Trailers are filled in for the message if they were requested.}
{\tt mach\_msg\_receive()} checks for {\tt MACH\_RCV\_TRAILER\_AV} or
{\tt MACH\_RCV\_TRAILER\_LABELS}.

\item {Message header is copied into the user's address space.}
{\tt ipc\_kmsg\_copyout()} calls {\tt ipc\_kmsg\_copyout\_header()}.
If the message contains ports rights they are copied via
{\tt ipc\_right\_copyout()} which first calls the appropriate Framework
entry point.  The entry points are {\tt mac\_check\_port\_hold\_send()}
for send rights, {\tt mac\_check\_port\_hold\_send\_once()} for send-once
rights, and {\tt mac\_check\_port\_hold\_receive()} for receive rights.

\item {Message body is copied into the user's address space.}
If the body contains ports rights (either inline or out of line)
they are copied in the same manner as rights contained in the
header.  If the body contains memory regions they are
mapped into the calling task's address space by the Mach
virtual memory subsystem.

\item {In-kernel message object is deallocated.}

\item {Message received.}

\end{enumerate}
