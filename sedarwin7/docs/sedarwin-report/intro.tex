\section{Introduction}

\subsection{SEDarwin}

Security-Enhanced Darwin (SEDarwin) is a port of the TrustedBSD
Mandatory Access Control (MAC) Framework and access control elements
derived from the NSA's Security Enhanced Linux (SELinux) from the
FreeBSD operating system to Apple's Mac OS X operating system.

Apple's Mac OS X operating system combines both open source and
proprietary technologies into a widely-used, production-quality
UNIX-based workstation supporting a broad variety of applications.

Darwin, the open source core of Apple's operating system, includes
a substantial quantity of code derived from the open source FreeBSD
operating system.  The TrustedBSD MAC Framework on FreeBSD is an
extensible security framework that allows developers to extend the
security properties of the system.  Based on the common source code
heritage, the SEDarwin project has ported the MAC Framework to run
on the Darwin platform, permitting the access control policies of
Darwin to be similarly extended.  Likewise, the SEBSD project, a
port of NSA's SELinux FLASK framework and Type Enforcement policy
language to FreeBSD, has been migrated to run on the Darwin platform.

This report provides background material relating to the component
technologies and research and development results of the migration
process. These include approaches for addressing differences between
the FreeBSD and Darwin MAC Framework implementations, techniques
for labeling Mach objects, application interface compatibility,
and the results of porting FLASK to the Darwin MAC Framework.

\subsection{Mac OS X, Darwin System Architectures}

Apple Computer's Mac OS X is a commercially available operating
system providing a graphical user interface on top of a UNIX core.
Darwin, the open source core of Mac OS X is composed of a variety
of technologies, including facilities derived from the Mach 3.0
microkernel, operating-system services based on FreeBSD 5, high
performance networking facilities, and multiple integrated file
systems.  Like most UNIX systems, the Darwin core is engineered for
stabilty, reliability, and performance.\cite{macosx,osxoverview}

Layered on top of the UNIX core are a collection of graphics routines,
a windowing system, and a user interface.  Together, these are the
components that help make Mac OS X a commercially viable system.

Darwin uses a Mach 3.0 core to manage processor resources (CPU and
memory), handle scheduling, to enforce memory protection, and to
implement interprocess communication.  Mach provides most of the
core OS services, including memory protection, preemptive multitasking,
and advanced virtual memory.  Unlike other systems based on Mach
3.0, the BSD kernel in Darwin runs in the kernel address space, not
as a (userspace) task.

A variant of 4.4BSD is used to support Mach and to provide networking
and file system access.  BSD components also provide process creation
and management, process signaling, system bootstrap and shutdown,
generic I/O routines, terminal and device handling, and resource
limits.

Darwin allows device drivers, networking extensions, and file systems
to be loaded dynamically.  Device drivers are created using the
IOKit object-oriented programming framework.  IOKit makes it easier
to develop plug and play modules for dynamic device management.
Since Darwin provides a binary interface for modular device drivers,
vendors are able to provide closed-source or proprietary modules
that work with the open source kernel.

Apple makes the Darwin source code available under the Apple Public
Source License, permitting third party inspection, modification,
and extension of the Darwin operating system.  This makes Darwin
an attractive platform for operating system research, since researchers
have the complete source code without the burden of non-disclosure
restrictions.  Through careful modification of the Darwin core,
operating system behavior can be changed while still allowing the
higher level application interfaces (Aqua, Quartz, Carbon, etc.)
to function, preserving the user experience.

\subsection{Mach}

Mach is a message-passing microkernel system. A running Mach system
consists of a number of tasks, each with its own address space, all
of which communicate with one another via messages. Tasks also
communicate with the kernel using messages--from the task's point
of view the kernel is just another task.

The object abstraction for messaging is the port right. A port right
is a handle to a message queue (the port) and a set of operations
(send or receive) that the task can perform on that port. In constrast
to Berkeley sockets, which have well-defined source and destination
ports, Mach ports are uni-directional. A port is only a destination
for a message.  As ports are relatively lightweight, applications
often use a port to represent a single higher-level object, such
as a window or hardware device.  Unlike many Unix IPC mechanisms,
while multiple tasks may hold a "send right" for a particular port,
only one task at a time (possibly the kernel) may receive messages
on a given port by holding a "receive right" for that port.

\subsection{Origin of Components}
\subsubsection{TrustedBSD MAC Framework on FreeBSD}

SPARTA ISSO (formerly McAfee Research) and the TrustedBSD Project
have implemented an extensible and modular kernel access control
framework permitting new access control policies to be introduced
into the FreeBSD kernel\cite{trustedbsd, bsdcon2000trustedbsd,
watson01, discex3, watson03}.  The TrustedBSD Mandatory Access
Control (MAC) Framework addresses many of the challenges associated
with introducing new access control services into operating system
kernels.  This is accomplished by abstracting common infrastructure
services from the policies, thus reducing the cost and complexity
of policy authoring.  The MAC Framework provides policy-independent
label storage for kernel objects, and persistent storage of labels
using file system extended attributes.  The MAC Framework composes
results from simultaneously loaded access control policies in a
predictable and reliable manner, permitting appropriately crafted
policies to be used in concert.

The MAC Framework augments the FreeBSD kernel to provide common
labeling infrastructure along with a set of entry points to intercept
operations on labeled objects.  The Framework supports labels on
file systems, processes, IPC, and network stack elements.  Each
registered policy may reserve space for security labels and implement
policy-specific behavior governing label content and use.  Labels
follow the kernel object life cycle and are initialized, allocated,
and destroyed along with their object.  Access control entry points
accept information about the action being performed, invoke each
registered policy, and compose the results into a success or failure.

\subsubsection{DTOS}

The Distributed Trusted Operating System (DTOS) project enhanced
the Mach 3.0 microkernel and developed a separate Mach security
server to help increase the security of process management, file
objects, and the network stack.  This project was a joint effort
by the NSA and Secure Computing Corporation (SCC).  This work
developed a prototype secure Mach microkernel that provided strong,
flexible security controls while minimizing the effects on performance
and application compatibility.\cite{dtos95,dtos96} The DTOS microkernel
made enhancements to the Mach 3.0 design, the same version of Mach
that Darwin is based upon.  DTOS used the Lites\cite{helander94unix} Mach
server as its BSD subsystem.  Unlike the BSD subsystem in Darwin,
Lites runs as a user-level single server; it is not kernel-resident
as in Darwin.

DTOS developed a separate security server to encapsulate their Type
Enforcement policy.  This allowed for a separation of the policy
decision (in the security server), and the the policy enforcement
(in the Mach microkernel).  This design ultimately led to the
development of FLASK and Security-Enhanced Linux.

While the DTOS code was not directly used in SEDarwin, it was
examined to help understand the complexities of introducing security
into a Mach-based operating system.  The goals of SEDarwin and DTOS
are quite similar: develop a prototype that can be used to demonstrate
that strong, useful security features can be introduced into an
operating system without sacrificing desirable features such as
performance or utility.

\subsubsection{FLASK}

The Flux Advanced Security Kernel (FLASK) is an operating system
security architecture that provides a flexible and fine-grained
mandatory access control (MAC) architecture\cite{SSLHAL99}.  It
was initially developed as a security-enhanced version of the Fluke
microkernel-based operating system\cite{ford96microkernels}.  FLASK
uses an architecture similar to DTOS, but with enhancements based
on lessons learned from the DTOS prototype.

The FLASK architecture provides flexibility by cleanly separating
the policy decision-making logic from the policy enforcement logic.
The policy decision-making logic is encapsulated within a single
component known as the security policy server.  Access control
decisions are enforced by a general framework used by the microkernel
as well as in object managers residing outside the microkernel.

A FLASK object manager binds a security identifier (SID) to active
kernel objects, and uses these SIDs as context during access control
checks.  The FLASK architecture includes an Access Vector Cache
(AVC) component that allows the object manager to cache access
decisions, reducing the performance penalty of the access checks.
The policy enforcement provides a source context, a target context,
a security class, and an access vector to the AVC to determine
access to an object.  Likewise, when an object manager wishes to
label a newly created object, it will consult the security server
to obtain a label.

\subsubsection{SELinux}

The National Security Agency's (NSA) Security-Enhanced
Linux\cite{loscocco01} (SELinux) is a version of the Linux
kernel that includes a port of the FLASK security server.

SELinux uses FLASK for the policy decision-making logic.  Policy
enforcement logic is implemented using the interfaces provided by
the Linux Security Modules framework\cite{loscocco01}.  A
wide range of security models can be implemented as security servers
without requiring changes to any other component of the system.
